% Copyright 2004 by Till Tantau <tantau@users.sourceforge.net>.
%
% In principle, this file can be redistributed and/or modified under
% the terms of the GNU Public License, version 2.
%
% However, this file is supposed to be a template to be modified
% for your own needs. For this reason, if you use this file as a
% template and not specifically distribute it as part of a another
% package/program, I grant the extra permission to freely copy and
% modify this file as you see fit and even to delete this copyright
% notice. 

\documentclass{beamer}

% There are many different themes available for Beamer. A comprehensive
% list with examples is given here:
% http://deic.uab.es/~iblanes/beamer_gallery/index_by_theme.html
% You can uncomment the themes below if you would like to use a different
% one:
%\usetheme{AnnArbor}
%\usetheme{Antibes}
%\usetheme{Bergen}
%\usetheme{Berkeley}
%\usetheme{Berlin}
%\usetheme{Boadilla}
%\usetheme{boxes}
%\usetheme{CambridgeUS}
%\usetheme{Copenhagen}
%\usetheme{Darmstadt}
%\usetheme{default}
%\usetheme{Frankfurt}
%\usetheme{Goettingen}
%\usetheme{Hannover}
%\usetheme{Ilmenau}
%\usetheme{JuanLesPins}
%\usetheme{Luebeck}
\usetheme{Madrid}
%\usetheme{Malmoe}
%\usetheme{Marburg}
%\usetheme{Montpellier}
%\usetheme{PaloAlto}
%\usetheme{Pittsburgh}
%\usetheme{Rochester}
%\usetheme{Singapore}
%\usetheme{Szeged}
%\usetheme{Warsaw}


% Customize Warsaw color 
\setbeamercolor*{palette primary}{use=structure,fg=white,bg=red!50!black}
\setbeamercolor*{palette secondary}{use=structure,fg=white,bg=red!60!black}
\setbeamercolor*{palette tertiary}{use=structure,fg=white,bg=red!70!black}

% Customize Warsaw block title and background colors
\setbeamercolor{block title}{bg=red!50!black,fg=white}


% List your packages here

\usepackage[colorinlistoftodos]{todonotes}
\usepackage{amsmath}

\title[]{A Generalized Open Source Platform Design for Building Energy Management}

% % A subtitle is optional and this may be deleted
% \subtitle{Product Proposal}

\author[B.~Lauer]{Brian~Lauer\\\and
Advisor: Dr. Suruz Miah}
% - Give the names in the same order as the appear in the paper.
% - Use the \inst{?} command only if the authors have different
%   affiliation.

\institute[Bradley University] % (optional, but mostly needed)
{
  Department of Electrical and Computer Engineering\\
  Bradley University\\
  1501 W. Bradley Avenue\\
  Peoria, IL, 61625, USA
}
% - Use the \inst command only if there are several affiliations.
% - Keep it simple, no one is interested in your street address.

\date[February~14,~2020]{Friday, February~14,~2020}
% - Either use conference name or its abbreviation.
% - Not really informative to the audience, more for people (including
%   yourself) who are reading the slides online

\logo{\hfill\href{http://www.bradley.edu}{\includegraphics[width=0.75cm]{../figs/logoBU1-Print}}}  % place logo in every page 


\subject{Mobile Robot Localization}
% This is only inserted into the PDF information catalog. Can be left
% out. 

% If you have a file called "university-logo-filename.xxx", where xxx
% is a graphic format that can be processed by latex or pdflatex,
% resp., then you can add a logo as follows:

% \pgfdeclareimage[height=0.5cm]{university-logo}{university-logo-filename}
% \logo{\pgfuseimage{university-logo}}

% Delete this, if you do not want the table of contents to pop up at
% the beginning of each subsection:
\AtBeginSubsection[]
{
  \begin{frame}<beamer>{Outline}
    \tableofcontents[currentsection,currentsubsection]
  \end{frame}
}




% Let's get started
\begin{document}

\begin{frame}
  \titlepage
\end{frame}

\begin{frame}{Outline}
  \tableofcontents
  % You might wish to add the option [pausesections]
\end{frame}

\section{Objectives}

\begin{frame}{Objectives}{}
\begin{itemize}
\item Implement microgrid state estimation algorithm in \cite{rana2018} in MATLAB
\item Research new device to add to BEMOSS
\end{itemize}
\end{frame}

\section{State Estimation Algorithm}
\begin{frame}{State Estimation Algorithm}{}
%\begin{figure}
%\includegraphics[scale=0.5]{../figs/img/microgridSchematic}
%\caption{Schematic of microgrid used in state estimation algorithm containing 2 Distributed Energy Resources (DERs)}
%\end{figure}
\end{frame}

\begin{frame}{State Estimation Algorithm}{}
By using KVL and KCL,
\[
\text{DER}i
\begin{cases}
\dot{V_{i,dq}}=-j \omega_0 V_{i,dq} + (d_i I_{ti,dq}+I_{ij,dq})/C_{ti}\\
\dot{I_{ti,dq}}=-j \omega_0 I_{t,dq} - (d_i V_{i,dq}+R_{ti,dq} I_{ti,dq} - V_{ti,dq})/L_{ti}\\
\end{cases}
\]
\[
\text{DER}j
\begin{cases}
\dot{V_{j,dq}}=-j \omega_0 V_{j,dq} + (d_j I_{tj,dq}+I_{ji,dq})/C_{tj}\\
\dot{I_{tj,dq}}=-j \omega_0 I_{tj,dq} -(d_j V_{j,dq} + R_{tj, dq} I_{tj,dq} - V_{tj,dq})/L_{tj}
\end{cases}
\]
\[
\dot{I_{ij,dq}}=-j \omega_0 I_{ij,dq} + (V_{j,dq} - R_{ij,dq} I_{ij,dq} - V_{i,dq})/L_{ij}
\]
\[
\dot{I_{ji,dq}}=-j \omega_0 I_{ji,dq} + (V_{i,dq} - R_{ji,dq} I_{ji,dq} - V_{j,dq})/L_{ji}
\]
\end{frame}
%\dot{I_{ti,dq}=-j \omega_0 I_{t,dq} - (d_i V_{i,dq}+R_{ti,dq} I_{ti,dq} - V_{ti,dq})/L_{ti}

\begin{frame}{State Estimation Algorithm}{}
In more compact form,
\[\dot{\boldsymbol{x}}=\boldsymbol{A}\boldsymbol{x}+\boldsymbol{B}\boldsymbol{u}\]
In discrete time,
\[\boldsymbol{x}(k+1)=\boldsymbol{A}_d \boldsymbol{x}(k)+\boldsymbol{B}_d \boldsymbol{u}(k) + \boldsymbol{n}_d(k)\]
where 
\[\boldsymbol{x}=[V_{i,d}, V_{i,q}, I_{ti,d}, I_{ti,q}, I_{ij,d}, I_{ij,q}, I_{ji,d}, I_{ji,q}, V_{j,d}, V_{j,q}, I_{tj,d}, I_{tj,q}]'\]
\[\boldsymbol{u}=[V_{ti,d}, V_{ti,q}, V_{tj,d}, V_{tj,q}]'\]
\[\boldsymbol{A}_d=\boldsymbol{I}+\boldsymbol{A}\Delta t \]
\[\boldsymbol{B}_d = \boldsymbol{B}\Delta t\]
and $\boldsymbol{A}$ and $\boldsymbol{B}$ are defined in \cite{rivers2014} and \cite{rana2018} respectively.
\end{frame}

\begin{frame}{State Estimation Algorithm}{}
Sensory measurements are made around the grid to estimate voltage.
\[\boldsymbol{y}^i(k)=\boldsymbol{C}^i \boldsymbol{x}(k) + \boldsymbol{w}^i(k)\]
where $i = 1,2,\ldots,n$. $\boldsymbol{y}^i(k)$ is the observations made, $\boldsymbol{C}^i$ is the sensing matrix, and $\boldsymbol{w}^i(k)$ is the measurement noise with zero mean and covariance $\boldsymbol{R}^i$.
\end{frame}

\begin{frame}{State Estimation Algorithm}{}
Distributed state estimation:
\begin{align*}
\hat{\boldsymbol{x}}^i(k+1) &= \boldsymbol{A}_d \hat{\boldsymbol{x}}^i(k)+\boldsymbol{B}_d \boldsymbol{u}(k)+\boldsymbol{K}^i(k)[\boldsymbol{y}^i(k)-\boldsymbol{C}^i \hat{\boldsymbol{x}}^i(k)]\\
&\quad + \boldsymbol{L}^i(k)\sum_{j\in N_i} [\hat{\boldsymbol{x}}^j(k)-\hat{\boldsymbol{x}}^i(k)]
\end{align*}
where $\boldsymbol{L}^i(k)$ is the neighboring gain and $\boldsymbol{K}^i(k)$ is the local gain.
\end{frame}

\section{Further tasks}
\begin{frame}{Further tasks}{}
\begin{itemize}
\item Research and start implementing new device
\item Start developing new BEMS from scratch
\end{itemize}
\end{frame}


% All of the following is optional and typically not needed. 
\appendix
\section<presentation>*{\appendixname}
\subsection<presentation>*{For Further Reading}

\begin{frame}[allowframebreaks]
  \frametitle<presentation>{For Further Reading}
    
  \begin{thebibliography}{10}
    
  \beamertemplatebookbibitems
  % Start with overview books.


 
    
  \beamertemplatearticlebibitems
  % Followed by interesting articles. Keep the list short. 

  \bibitem{rivers2014}
    S. Riverso, F. Sarzo, and G. Ferrari-Trecate
    \newblock Plug-and-play voltage and frequency control of islanded
microgrids with meshed topology	
    \newblock IEEE Transactions on Smart Grid, vol. 6, no. 3, pp. 1176-1184, May 2015.
    
  \bibitem{rana2018}
    M.M. Rana, L. Li, S. W. and W. Xiang
    \newblock {\em Consensus-Based Smart Grid State Estimation Algorithm}.
    \newblock IEEE Transactions on Industrial Informatics, vol. 14, no. 8, pp. 3368-3375, Aug. 2018.
  \end{thebibliography}
\end{frame}

\begin{frame}
\Huge
\center

Any questions?
\end{frame}

\end{document}



%%% Local Variables:
%%% mode: latex
%%% TeX-master: t
%%% End:
