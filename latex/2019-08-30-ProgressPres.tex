% Copyright 2004 by Till Tantau <tantau@users.sourceforge.net>.
%
% In principle, this file can be redistributed and/or modified under
% the terms of the GNU Public License, version 2.
%
% However, this file is supposed to be a template to be modified
% for your own needs. For this reason, if you use this file as a
% template and not specifically distribute it as part of a another
% package/program, I grant the extra permission to freely copy and
% modify this file as you see fit and even to delete this copyright
% notice. 

\documentclass{beamer}

% There are many different themes available for Beamer. A comprehensive
% list with examples is given here:
% http://deic.uab.es/~iblanes/beamer_gallery/index_by_theme.html
% You can uncomment the themes below if you would like to use a different
% one:
%\usetheme{AnnArbor}
%\usetheme{Antibes}
%\usetheme{Bergen}
%\usetheme{Berkeley}
%\usetheme{Berlin}
%\usetheme{Boadilla}
%\usetheme{boxes}
%\usetheme{CambridgeUS}
%\usetheme{Copenhagen}
%\usetheme{Darmstadt}
%\usetheme{default}
%\usetheme{Frankfurt}
%\usetheme{Goettingen}
%\usetheme{Hannover}
%\usetheme{Ilmenau}
%\usetheme{JuanLesPins}
%\usetheme{Luebeck}
\usetheme{Madrid}
%\usetheme{Malmoe}
%\usetheme{Marburg}
%\usetheme{Montpellier}
%\usetheme{PaloAlto}
%\usetheme{Pittsburgh}
%\usetheme{Rochester}
%\usetheme{Singapore}
%\usetheme{Szeged}
%\usetheme{Warsaw}


% Customize Warsaw color 
\setbeamercolor*{palette primary}{use=structure,fg=white,bg=red!50!black}
\setbeamercolor*{palette secondary}{use=structure,fg=white,bg=red!60!black}
\setbeamercolor*{palette tertiary}{use=structure,fg=white,bg=red!70!black}

% Customize Warsaw block title and background colors
\setbeamercolor{block title}{bg=red!50!black,fg=white}


% List your packages here

\usepackage[colorinlistoftodos]{todonotes}


\title[Progress Update]{A Universal Platform for Building Energy Management}

% % A subtitle is optional and this may be deleted
% \subtitle{Product Proposal}

\author[B.~Lauer]{Brian~Lauer \\\and
Advisor: Dr. Suruz Miah}
% - Give the names in the same order as the appear in the paper.
% - Use the \inst{?} command only if the authors have different
%   affiliation.

\institute[Bradley University] % (optional, but mostly needed)
{
  Department of Electrical and Computer Engineering\\
  Bradley University\\
  1501 W. Bradley Avenue\\
  Peoria, IL, 61625, USA
}
% - Use the \inst command only if there are several affiliations.
% - Keep it simple, no one is interested in your street address.

\date[August~30,~2019]{Friday, August~30,~2019}
% - Either use conference name or its abbreviation.
% - Not really informative to the audience, more for people (including
%   yourself) who are reading the slides online

\logo{\hfill\href{http://www.bradley.edu}{\includegraphics[width=0.75cm]{../figs/logoBU1-Print}}}  % place logo in every page 


\subject{Mobile Robot Localization}
% This is only inserted into the PDF information catalog. Can be left
% out. 

% If you have a file called "university-logo-filename.xxx", where xxx
% is a graphic format that can be processed by latex or pdflatex,
% resp., then you can add a logo as follows:

% \pgfdeclareimage[height=0.5cm]{university-logo}{university-logo-filename}
% \logo{\pgfuseimage{university-logo}}

% Delete this, if you do not want the table of contents to pop up at
% the beginning of each subsection:
% Outline
% Existing BEM solutions
% 


% Let's get started
\begin{document}

\begin{frame}
  \titlepage
\end{frame}

\begin{frame}{Outline}
  \tableofcontents
  % You might wish to add the option [pausesections]
\end{frame}

\section{Existing BEM Solutions}
\subsection{BEMServer}

\begin{frame}{Existing BEM Solutions}{BEMServer}
%\begin{figure}
%\includegraphics[scale=0.22]{../figs/img/modules-store.jpg}
%\caption{Image from \cite{nobatek2019}}
%\end{figure}
\end{frame}

\subsection{BuildingOS Facilities}

\begin{frame}{Existing BEM Solutions}{BuildingOS Facilities}
%\begin{figure}
%\includegraphics[scale=0.3]{../figs/img/whatwedo-efficiency.jpg}
%\caption{Image from \cite{lucid2019}}
%\end{figure}
\end{frame}

\section{Agenda}
\begin{frame}{Agenda}
\begin{block}{Paper}
\begin{itemize}
	\item Complete sections
	\item Complete Gantt chart
\end{itemize}
\end{block}
\end{frame}

\section{Kill A Watt}

\begin{frame}{Kill A Watt}
%\begin{figure}
%\includegraphics[scale=0.32]{../figs/img/main_p4400.jpg}
%\caption{Image from \cite{p32019}}
%\end{figure}
\end{frame}

\begin{frame}{Kill A Watt}
%\begin{figure}
%\includegraphics[scale=0.3]{../figs/img/kaw3_1-1.pdf}
%\caption{Image from \cite{compediumarcana}}
%\end{figure}
\end{frame}

\begin{frame}{Kill A Watt}
%\begin{figure}
%\includegraphics[scale=0.33]{../figs/img/kaw3_2-end.pdf}
%\caption{Image from \cite{compediumarcana}}
%\end{figure}
\end{frame}

%\begin{frame}{Blocks}
%\begin{block}{Block Title}
%You can also highlight sections of your %presentation in a block, with it's own %title
%\end{block}
%\begin{theorem}
%There are separate environments for %theorems, examples, definitions and proofs.
%\end{theorem}
%\begin{example}
%Here is an example of an example block.
%\end{example}
%\end{frame}

% Placing a * after \section means it will not show in the
% outline or table of contents.
% All of the following is optional and typically not needed. 
\appendix
\section<presentation>*{\appendixname}
\subsection<presentation>*{For Further Reading}

\begin{frame}[allowframebreaks]
  \frametitle<presentation>{For Further Reading}
    
  \begin{thebibliography}{10}
    
  \setbeamertemplate{bibliography item}[online]
  % Start with overview books.

  \bibitem{p32019}
    \newblock {\em P4400 Kill A Watt}.
    \newblock \texttt{http://www.p3international.com/products/p4400.html}.
    
  \bibitem{nobatek2019}
  	\newblock {\em BEMServer}.
  	\newblock \texttt{https://www.bemserver.org/}
  	
  \bibitem{lucid2019}
  	\newblock {\em BuildingOS}.
  	\newblock \texttt{http://buildingos.com/about/}

  \bibitem{compediumarcana}
  	\newblock {\em compediumarcana}.
  	\newblock \texttt{http://compendiumarcana.com/kaw/}

  \end{thebibliography}
  
\end{frame}

\end{document}



%%% Local Variables:
%%% mode: latex
%%% TeX-master: t
%%% End: